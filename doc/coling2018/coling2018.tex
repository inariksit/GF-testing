%
% File coling2018.tex
%
% Contact: zhu2048@gmail.com & liuzy@tsinghua.edu.cn
%% Based on the style files for COLING-2016, which were, in turn,
%% Based on the style files for COLING-2014, which were, in turn,
%% Based on the style files for ACL-2014, which were, in turn,
%% Based on the style files for ACL-2013, which were, in turn,
%% Based on the style files for ACL-2012, which were, in turn,
%% based on the style files for ACL-2011, which were, in turn, 
%% based on the style files for ACL-2010, which were, in turn, 
%% based on the style files for ACL-IJCNLP-2009, which were, in turn,
%% based on the style files for EACL-2009 and IJCNLP-2008...

%% Based on the style files for EACL 2006 by 
%%e.agirre@ehu.es or Sergi.Balari@uab.es
%% and that of ACL 08 by Joakim Nivre and Noah Smith

\documentclass[11pt]{article}
\usepackage{coling2018}
\usepackage{times}
\usepackage{url}
\usepackage{color}
\usepackage{latexsym}
\usepackage{amssymb}
\usepackage{stackengine}

\def\t#1{\texttt{#1}}
\newcommand{\tts}[1]{{\tt #1}}
%\newcommand{\tts}[1]{{\tt \small #1}}
\newcommand{\todo}[1]{{\color{cyan}\textbf{[TODO: }#1\textbf{]}}}

%\setlength\titlebox{5cm}

% You can expand the titlebox if you need extra space
% to show all the authors. Please do not make the titlebox
% smaller than 5cm (the original size); we will check this
% in the camera-ready version and ask you to change it back.


\title{Automatic test case generation for multilingual grammars}

\author{Inari Listenmaa and Koen Claessen \\
  Department of Computer Science and Engineering \\
  University of Gothenburg and Chalmers University of Technology \\
  Gothenburg, Sweden \\
  {\tt inari@chalmers.se, koen@chalmers.se} }
%\\\And
%  Koen Claessen \\
%  Department of Computer Science and Engineering \\
%  Chalmers University of Technology \\
%  Gothenburg, Sweden \\
 % {\tt koen@chalmers.se} \\}

\date{}

\begin{document}
\maketitle
\begin{abstract}
  We present a method to generate a set of phrases from a multilingual
  grammar, intended for a human oracle to look at. This requires a
  minimal and representative set of sentences, such that all
  constructions from the grammar are tested with relevant words.
\end{abstract}

\section{Introduction}
Grammatical Framework (GF) \cite{ranta2011gfbook} 
is a framework for building multilingual grammar applications. Its main
components are a functional programming language for writing grammars
and a resource library that contains the linguistic details of many
natural languages.
A GF program consists of an \emph{abstract syntax} (a set of functions
and their categories) and a set of one or more
\emph{concrete syntaxes} which describe how the abstract
functions and categories are linearized (turned into surface strings) in each
respective concrete language. The resulting grammar
describes a mapping between concrete language strings and
their corresponding abstract trees (structures of function names).
This mapping is bidirectional---strings can be \emph{parsed} to
trees, and trees \emph{linearised} to strings.
As an abstract syntax can have multiple corresponding concrete syntaxes,
the respective languages can be automatically \emph{translated} from one to the other by
first parsing a string into a tree and then linearising the obtained tree
into a new string.

\todo{We test only what the grammar outputs, not positive and negative
  (sentences that can be parsed / cannot be parsed)}

\section{The problem}

Traditionally, GF grammars are tested by the grammarians themselves,
much like unit testing. When implementing some feature, such as 
relative clauses, the grammarian comes up with a test suite of 
sentences that include relative clauses, and stores in the form of 
abstract syntax trees. In principle, a test suite created for one 
language can easily be reused for another, because the ASTs are 
identical. Ideally, every time someone touches relative clauses 
in the any concrete syntax, the trees in the test suite will be 
linearised with the changed concrete syntax, and verified by someone
who speaks the language (or compared to the original gold standard, 
if there is one). This scheme can fail for various reasons: 

\begin{itemize}
\item The original list is not exhaustive: for instance, it tests only
``X, who loves me'' but not ``X, whom I love''. 
\item The original list is exhaustive in one language, but not in all:
for instance, it started in English and only included one noun, but in
French it would need at least one masculine and one feminine noun. 
\item The list is overly long, with redundant test cases, and human
testers are not motivated to read through. 
\item A grammarian makes a change somewhere else in the grammar, and
does not realize that it affects relative clauses, and thus does not
rerun the appropriate test suite. 
\end{itemize}


\section{Experiences as a grammar writer}


GF resource grammars include basic morphology and syntax of a
language. GF morphology and lexicon can often be extracted fully or
partly from data or existing morphological lexica, but the development
of syntactic functions requires intensive human labour, and advanced
knowledge on both linguistics and programming. 
Many GF resource grammars are written by non-native speakers, relying
on a grammar book and possibly a native informant. 

%To make the problem more concrete, here is a set of problems, all found in real GF resource grammars.

%\begin{itemize}
%\item Prepositions should merge with a single determiner, e.g. \emph{with~this} $\rightarrow$ \emph{thiswith}, but stay independent when the determiner quantifies a noun, e.g. \emph{with this house}.
%\item Word order 
%\end{itemize}
%Concrete problems: the kind of bugs we set out to find
%* Empty categories
%* Empty fields
%* Eliminated arguments

%Pick ~3 bugs: 
%what tree was generated? why? likelihood of finding this bug without our method.

%Results


\begin{figure}[h]
  \centering
    \begin{verbatim}
abstract NounPhrases = {
  flags startcat = NP ;
  cat
    NP ; Adv ;                   -- Categories formed from other categories
    CN ; Det ; Adj ; Prep ;      -- Lexical categories
  fun
    DetNP : Det -> NP ;          -- e.g. "this"
    DetCN : Det -> CN -> NP ;    -- e.g. "this house"
    PrepNP : Prep -> NP -> Adv ; -- e.g. "without this house"
    AdjCN : Adj -> CN -> CN ;    -- e.g. "small house"
    AdvCN : Adv -> CN -> CN ;    -- e.g. "house on this hill"

    house, hill, cake : CN ;
    this, these, your : Det ;
    good, small, blue : Adj ;
    on, with, without : Prep ;
}
\end{verbatim}
  \caption{GF grammar for noun phrases}
\label{fig:exampleGrammar}
\end{figure}

Figure~\ref{fig:exampleGrammar} shows a small example of a GF
grammar. The grammar generates noun phrases for a lexicon of eight
words (\emph{house, hill, \dots, without}) in four lexical categories,
and five functions to construct phrases.  
\t{CN} stands for common noun, and it can be modified by arbitrarily
many adjectives (\t{Adj}), e.g. \emph{small blue house}. A \t{CN} is
quantified into a noun phrase (\t{NP}) by adding a determiner
(\t{Det}); alternatively, a \t{Det} can also become an independent
noun phrase. Finally, we can form an adverb (\t{Adv}) by combining a
preposition (\t{Prep}) with an \t{NP}, and those adverbs can modify
yet another \t{CN}. 
We refer to this grammar throughout the paper.

\subsection{How it works}

\paragraph{Test cases} 
The basic unit of a test case is a single constructor, and the first step 
is to build a set of trees using that constructor.
If we are interested in a single lexical item, such as \emph{small}, 
then the subtree \t{small} is the full set of trees. If we choose a function 
with arguments, such as \t{PrepNP}, then we do the following: 
\begin{itemize}
\item For each argument type (\t{Prep} and \t{NP}), compute the
  set of minimal and representative trees. This is a recursive
  process: to compute the set of trees in \t{NP}, we must consider
  all functions that create its argument types (\t{Det} and \t{CN}),
  until we have a set of lexical functions to choose from. 
\item Take a cross product, prune out redundant combinations, and
  apply the constructor \t{PrepNP} to the resulting set of
  arguments. The pruning method will be described later. 
\end{itemize}

\paragraph{Test cases using \t{AdjCN}} Let us test the function
\t{AdjCN : Adj $\rightarrow$ CN $\rightarrow$ CN}, and take a Spanish
concrete syntax as an example. 
Firstly, we need a minimal and representative set of arguments of types
\t{Adj} and \t{CN}. Consider the nouns first: Spanish has a
grammatical gender, so in order to be representative, we need an
example of both masculine and feminine. Out of the small lexicon,
\t{house} (\emph{casa}) is feminine, and \t{hill} (\emph{cerro}) is
masculine, so we return those two nouns as the full set of argument
trees in \t{CN}. 

Secondly, we consider the adjectives. Most commonly, adjectives follow
the noun, e.g. \emph{casa peque\~{n}a} `small house', but some
adjectives precede the noun, e.g. \emph{buena casa} `good house'. Thus 
in order to cover the full spectrum of adjective placement, we need
one premodifier and one postmodifier adjective. We pick the words
\t{good} and \t{small} as the arguments of type \t{Adj}. 

Now, our full set of test cases are \t{AdjCN} applied to the cross
product of \{\stackanchor{\tt \small good}{\tt \small small}\}
$\times$ \{ \stackanchor{\tt \small house}{\tt \small hill}\}.
Note that \t{CN} is unspecified for number, because it is still waiting for a
determiner (e.g. \t{this} or \t{these}) to complete it into an
\t{NP}. Thus all the test cases contain both singular and plural
variants, and by linearizing these 4 trees, we get the 8 strings shown
in Figure~\ref{fig:adjAttr}.

\begin{figure}
\centering
\begin{minipage}{.5\textwidth}
\centering
\begin{tabular}{| l | l |}
\hline
\t{AdjCN good house}   & \t{AdjCN good hill} \\ 
\textsc{(sg)} buena casa             & \textsc{(sg)} buen cerro \\
\textsc{(pl)} buenas casas           & \textsc{(pl)} buenos cerros \\ \hline

\t{AdjCN small house}   & \t{AdjCN small hill} \\ 
\textsc{(sg)} casa  peque\~{n}a            & \textsc{(sg)} cerro  peque\~{n}o \\
\textsc{(pl)} casas  peque\~{n}as          & \textsc{(pl)} cerros  peque\~{n}os \\ \hline
\end{tabular}
\caption{Agreement and placement of adjectives in attributive position}
\label{fig:adjAttr}
\end{minipage}%
\begin{minipage}{.5\textwidth}
  \centering
  \todo{A picture of trees with holes}
 \caption{Trees with a hole of type \t{CN}}
\label{fig:treesWithHoles}
\end{minipage}
\end{figure}


\paragraph{Context} 

Now that we have a set of test cases, we create contexts that show all
variation within them. By \emph{context}, we mean simply a tree in
some other category, with a \emph{hole} of type \t{CN}, as shown in
Figure~\ref{fig:treesWithHoles}. 
In this grammar, the only category above is \t{NP}, and there is only
one way that a \t{CN} can end up in an \t{NP}: by using the function
\t{DetCN : Det $\rightarrow$ CN $\rightarrow$ NP}. 

Just like before, we start by creating a representative and minimal
set of arguments of type \t{Det}. So far the relevant features have
been grammatical gender an adjective placement---we have 4 trees with
all combinations of \{masculine, feminine\} and \{pre,post\}. The
resulting trees have \emph{variable} number, but now we have a
category \t{Det} with an inherent number, so we can create the context
that picks the string ``casa  peque\~{n}a'', and another that picks
``casas  peque\~{n}as''. By simple combinatorics, we pick the
representative set of \t{Det} with one singular and one plural
element, e.g. \t{this} and \t{these}, and form two contexts:
\verb|DetCN this _| and  \verb|DetCN these _|. We insert the 4 test
cases into the holes, and get 8 trees in total: \t{\{DetCN this (good
  house), \dots, DetCN these (small hill)\}}. 


The grammar, with only 12-word lexicon and 5 syntactic functions,
generates \todo{n} trees up to depth 5\footnote{example of a tree of that depth}. 
These 8 test cases are enough to test whether the
syntactic function \t{AdjCN} is correctly implemented.   

Of course, this test set will not catch if e.g. any of the lexical
functions contains a typo, or if there is a more severe bug in the
morphological paradigms. But there are easier methods to test for such
bugs---our goal is to test the more abstract, syntactic phenomena with
as few trees as possible. 
%a bug in the morphological paradigms will go unnoticed, 


\paragraph{Pruning}

For the previous example, we did not need any pruning: the cross
product of all relevant distinctions produced a minimal and
representative set of trees. Now assume we have a larger grammar,
which also covers adjectives in a predicative position:
e.g. \emph{esta casa es peque\~{n}a} `this house is small' and
\emph{esta casa es buena} `this house is good'. The distinction which
made us choose \t{small} and \t{good} in the first place is now gone:
in predicative position, both adjectives behave the same. Thus, in
this larger grammar, when we want to test adjectives, it is necessary
to include two examples when in attributive position, but only one
when in predicative position. We describe exactly how this works in
Section~\ref{sec:details}. 
%In total we would get 12 sentences: the 8 shown previously, and 4 

\subsection{Examples of representative and minimal test cases} 
To make the example more concrete, we list out some phenomena found in
real grammars, and how to test their implementation. 

\paragraph{Preposition contraction in Dutch} Some prepositions should
merge with a single determiner or pronoun, e.g. \emph{met~deze} `with
this' becomes \emph{hiermee} `herewith', but stay independent when the
determiner quantifies a noun, e.g. \emph{met deze huis} `with this
house'. When testing \t{PrepNP}, the program should generate one
preposition that contracts and one that does not, as well as one
\t{NP} that is a single determiner, and one that comes from a noun.
In order to catch a bug in the function, or confirm there is none, the
program should generate the four trees 
\t{PrepNP} \{ \stackanchor{\tt with}{\tt without} \} 
           \{ \stackanchor{\tt DetNP this}{\tt DetCN this house} \}. 

\paragraph{Adjective agreement in Estonian} In attributive position,
most adjectives agree with nouns in case and number. However,
participle adjectives are invariable (singular nominative) as
attributes but inflect regularly in a predicative position, and a set
of invariable adjectives do not inflect in any position. Furthermore,
in 4 of the 14 grammatical cases, even the regular adjectives only
agree with the noun in number, but the case is always genitive: e.g. 
\stackanchor{ \emph{suur-e} \emph{auto-ga}}{\small \emph{big}-\textsc{gen} \emph{car}-\textsc{com}} 
%\emph{suur-e} \emph{auto-ga}  \emph{big}-\textsc{gen} \emph{car}-\textsc{com} 
`with a big car'. 
Thus in order to test adjectives as attributes, we need an example for
one of the 10 ``usual'' cases and one of the 4 ``unusual'' cases, one
of each type of adjective, and one of each number.  
The following test set creates all the relevant cases:
 \t{PrepNP} \{ \stackanchor{\tt on}{\tt with} \}
             {\tt (DetCN} \{ \stackanchor{\tt this}{\tt these} \} 
             {\tt AdjCN}  \{ \stackanchor{\stackanchor{\tt }{\tt
                 small}}{\stackanchor{\tt tired}{\tt ready}} \} 
             {\tt house)}

%To test the whole phenomenon, we would need one more context, where
%the adjective is in the predicative position `the house is \verb|_|',
%and one more example adjective---as an attribute, we can lump
%participles and invariables into one test example, but as a
%predicative, regular and participles behave the same and differently
%from invariable.


\paragraph{Determiner placement in Basque} When a number (other than 1) or a
possessive pronoun acts as a determiner in a complex noun phrase, it
is placed between ``heavy'' modifiers, such as adverbials or relative clauses,
and the rest of the noun phrase. Demonstrative pronouns, such as \emph{this}, 
 are placed after all modifiers as an independent word. Number 1,
 which functions as an indefinite article, acts like demonstratives, but
 the definite article is a suffix. If there is a ``light'' modifier,
 such as an adjective, the definite article attaches to the modifier;
 otherwise it attaches to the noun. In order to test the
 implementation of this phenomenon, we need a
subset of the following trees: \\ \\ 
\t{DetCN} \{
\stackanchor{\stackanchor{\tts{a}}{\tts{the}}}{\stackanchor{\tts{this}}{\tts{your}}}
\} \{ \stackanchor{\tt AdvCN on\_the\_hill}{$\varnothing$} \} 
\{ \stackanchor{\tt AdjCN small}{$\varnothing$} \}  {\tt house} 



\section{Technical details}
\label{sec:details}

GF grammar compiles into a low-level format called PGF. After the
compilation, we get one category for each combination of parameters:
for English adjectives, \texttt{A => A$_{pos}$, A$_{comp}$,
A$_{superl}$}, and for Spanish, \texttt{A => A$_{pos×sg×masc}$, \dots,
A$_{superl×pl×fem}$}. 

Suddenly, we have a bunch of new types, and those are different for
each concrete syntax! The original question ``we need a sample of
nouns/verbs/… that makes sense'' can be simplified ``we need one
noun/verb/… of each type''. The types are determined by the parameters
in the concrete syntax. 

So remember all the hassle when you can't pattern match strings to
know something, but instead you have to define a parameter? This is
actually a nice side effect from that: each parameter contributes to a
new category, so it pays off in generating examples. If the feature is
important for your grammar---say that in language A, negation is
simply attaching the word  ``no'' before the verb, and in language B,
negation changes the word order and the object case. Then in the GF
grammar for language B, we would need a Boolean \texttt{isNeg} field
in the relevant categories, which we then pattern match against in
order to determine the relevant operations. That parameter in the
abstract category translates into different concrete categories, and
that way, when we generate example trees, we make sure to include one
of each. For instance, in language A, we could end up with the trees
``any horse'' and ``all horses'' when testing NPs, but in language B,
the set would also include ``no horses''. 



\section{Evaluation}

How many trees you eliminate -- how many trees would you have to read
to confirm that you're correct, and how many trees you need to read
now

We've used the same treebank for all languages
* Janna's thesis, treebank for Russian (paper in ACL coling 2005)
* UD treebank, Penn treebank

Check what HPSG, TAG, CCG etc. people use for eval
Stephan Müller's sentence list testing only weak generative capacity,
this method is testing strong generative capacity


\begin{itemize}
\item Cost
  \begin{itemize}
  \item time of generating examples
  \item how many examples generated -- analysis if they're redundant
  \item time of looking at examples
  \end{itemize}

\item Effect
  \begin{itemize}
  \item compare against other methods -- what methods?
  \item For application grammars, if you're writing them from scratch, it is actually pretty feasible to just gt the hell out of it as you write. But this doesn't work for bigger grammars.
  \item Morphology can be tested efficiently againts any existing morphological analyser. I've used Apertium for Dutch and Basque.
  \end{itemize}
\end{itemize}


\section{Discussion}

We have only concentrated on GF grammars so far, but the method works
for any grammar formalism that can be compiled into parallel
multiple CFG (PMCFG) \todo{cite}. The expressive power of PMCFG lies
between mildly context-sensitive and context-sensitive, and thus any
grammar formalism that is less expressive can be expressed as a
PMCFG. This includes CFGs, and mildly context-sensitive formalisms
such as TAG and CCG \todo{cite}. GF already supports reading grammars
in labeled BNF format; so testing any existing CFG is a matter of some
pre-processing.

Some existing formalisms such as HPSG ans LFG are fully
context-sensitive \todo{cite and find out if there's some subset with
  nice properties}.


Theoretical question: two grammars that produce the same language but
have different grammar rules.
NounPhrasesEus and NounPhrasesEusBind -- same number of concrete
categories??? \todo{maybe more interesting statistics to show!}

Language typological implications -- or just grammar engineering implications,
how many parameters?


We plan to look into existing text corpora, and find trees that are
structurally identical  to those that our program generates as a
minimal and representative example. As a simplified example, ``a worm
without winter'', generated by the program, would be identical\footnote{This particular example holds for English, but in another language, the words ``pizza'' and ``worm'', as well as ``winter'' and ``cheese'' may not match in all relevant features---grammatical gender, whether the word starts with a vowel or a consonant, etc. All this information comes from the concrete syntax!} 
in structure to ``a pizza without cheese'', found in a real text, and
can thus be substituted for the generated one.   
Alternatively, we could use statistical information on co-occurrences
of words, and generate appropriate pools of words, from which we draw
example sentences. 

% include your own bib file like this:
\bibliographystyle{acl}
\bibliography{bibliography}

\end{document}
