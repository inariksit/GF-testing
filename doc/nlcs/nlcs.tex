% This is samplepaper.tex, a sample chapter demonstrating the
% LLNCS macro package for Springer Computer Science proceedings;
% Version 2.20 of 2017/10/04
%
\documentclass[runningheads]{llncs}
%
\usepackage{graphicx}
% Used for displaying a sample figure. If possible, figure files should
% be included in EPS format.
%
% If you use the hyperref package, please uncomment the following line
% to display URLs in blue roman font according to Springer's eBook style:
% \renewcommand\UrlFont{\color{blue}\rmfamily}
\usepackage{url}
\usepackage{color}
\usepackage{stackengine}
\usepackage[normalem]{ulem}

\def\t#1{\texttt{#1}}
\def\gf{\textsc{gf}}
\def\pgf{\textsc{pgf}}
\def\lfg{\textsc{lfg}}
\def\ccg{\textsc{ccg}}
\def\tag{\textsc{tag}}
\def\cfg{\textsc{cfg}}
\def\pmcfg{\textsc{pmcfg}}
\def\hpsg{\textsc{hpsg}}
\newcommand{\quality}[1]{${\tt AP_{#1}}$}
\newcommand{\kind}[1]{${\tt CN_{#1}}$}
\newcommand{\very}[1]{${\tt Very_{#1}}$}
\newcommand{\comment}{${\tt S}$}
\newcommand{\mod}[2]{${\tt Mod_{#1\times#2}}$}
\newcommand{\pred}[3]{${\tt Pred_{#1\times#2\times#3}}$}
\newcommand{\itemSpa}[2]{${\tt NP_{#1\times#2}}$}
\newcommand{\itemEng}[1]{${\tt NP_{#1}}$}
\newcommand{\todo}[1]{{\color{cyan}\textbf{[TODO: }#1\textbf{]}}}

\begin{document}
%
\title{Automatic test suite generation for PMCFG grammars}
%
%\titlerunning{Abbreviated paper title}
% If the paper title is too long for the running head, you can set
% an abbreviated paper title here
%
\author{Inari Listenmaa \and
Koen Claessen}
%
\authorrunning{I. Listenmaa and K. Claessen}
% First names are abbreviated in the running head.
% If there are more than two authors, 'et al.' is used.
%
\institute{Department of Computer Science and Engineering \\
 University of Gothenburg and Chalmers University of Technology \\
 Gothenburg, Sweden.\\
\{inari,koen\}@chalmers.se}
%\url{http://http://www.cse.chalmers.se/}
%
\maketitle              % typeset the header of the contribution
%
\begin{abstract}
The abstract should briefly summarize the contents of the paper in
15--250 words.

\keywords{First keyword  \and Second keyword \and Another keyword.}
\end{abstract}
%
%
%
\section{Introduction}

\todo{Rewrite this whole bit until the example; this is just a placeholder}
What is the essence of a language? 
Suppose we have a potentially infinite grammar, and we want to find a
minimal and representative set of sentences, along with their
analyses, to test the correctness of the grammar.

In general computer science, \emph{kernelization} \todo{cite} is used
to preprocess inputs given to some expensive algorithm: for instance,
a large graph can be reduced into a smaller graph, which produces the
same output, but runs much faster than the original large graph. We
propose a similar technique for grammars, with the purpose of building
test suites of trees. These test suites could be given for human
oracles to read (which is an extremely expensive algorithm!), they
could be tested against a large corpus, or a combination of the two. 

\todo{Human judgment is inavoidable, but we want to minimize the
  number of sentences the human reads.}

To give an intuitive example: in English we need to test a reflexive
construct with three different 3rd person singular subjects, because
the object has to agree with the subject: ``he sees himself'', ``she
sees herself'' and ``it sees itself''. Without seeing all three
examples, we cannot be certain that the reflexive construction is
implemented correctly. In contrast, the general pattern of a transitive
verb with a non-reflexive object is enough to test with only one third
person subject: \emph{he, she, it}, or any singular noun or proper
name. The agreement only shows in the verb form, thus including both
``she sees a dog'' and ``John sees a dog'' in the test suite is redundant. 

Now, what is minimal and representative is highly language-dependent. 
For instance, Basque transitive verbs agree with both subject and
object, thus we need 6 $\times$ 6 examples just to cover all verb
forms. We are not interested in the morphology per se---there are
easier methods to test for that---but the correctness of the syntactic
function: does the function pick the correct verb form for the correct
combination of subject and object? For that purpose, it is enough to
test the syntactic construction ``transitive verb phrase'' with just a
single transitive verb.

Our concrete implementation is for a particular grammar formalism,
namely parallel multiple context-free grammars ({\sc pmcfg})
\cite{seki91pmcfg}, which is the core formalism used by the
Grammatical Framework (\gf) \cite{ranta2004gf}. However, the general
method works for any formalism that is at most as expressive as
\pmcfg{}, including formalisms such as Tree-Adjoining Grammar (\tag)
\cite{joshi1975tag} and several variants of Categorial Grammars
\todo{cite about the fragment of ACG that is MCFG}.
Combinatorial Categorial Grammar (\ccg) \cite{steedman1988ccg}.

\section{Grammatical Framework}
\label{sec:GF}

\todo{General \gf{} introduction.\\
1) Emphasize that analysis and generation are all in the same
  grammar. \\ 2) GF as a formalism can be as syntactic or semantic as one wants.
}

\begin{figure}[h]
  \caption{GF grammar about food}
\label{fig:exampleGrammar}
\centering
    \begin{verbatim}
abstract Foods = {
  flags startcat = S ;
  cat
    S ; NP ; CN ; AP ;
  fun
    Pred : NP -> AP -> S ;
    This, That, These, Those : CN -> NP ;
    Mod : AP -> CN -> CN ;
    Wine, Cheese, Fish, Pizza : CN ;
    Very : AP -> AP ;
    Warm, Good, Italian, Vegan : AP ;
}
    \end{verbatim}
\end{figure}

Figure~\ref{fig:exampleGrammar} shows a small example of a GF
grammar. We refer to this grammar throughout sections~\ref{sec:GF}--\ref{sec:testing}. 

\t{CN}, `common noun', can be modified by adjectives or adverbs, but
it hasn't yet been quantified into a \t{NP}. 


\section{PMCFG}
\label{sec:PMCFG}

\gf{} grammars are compiled into parallel multiple context-free
grammars (\pmcfg), which are processed by our tool. Here we explain
three key features that are important for the test suite generation.

\subsubsection{Concrete categories}

For each category in the original grammar, the \gf{} compiler
introduces a new category in the \pmcfg{} for each combination of
inherent parameters.  
% For example, the English \t{NP} has an inherent property of number,
% so it compiles into two \pmcfg{} categories: \itemEng{sg} and
% \itemEng{pl}. Spanish noun phrases also have an inherent property of
% gender, so the same abstract category of \t{NP} compiles into four
% concrete categories: \{\itemSpa{sg}{masc}, \dots, \itemSpa{pl}{fem}\}.
These categories can be linearized to strings or vectors of
strings. The start category (\t{S} in the Foods grammar) is in
general a single string, but intermediate categories may have to keep
several options open. 

Consider the categories \t{NP}, \t{CN} and \t{AP} in the
Spanish concrete syntax. Firstly, \t{NP} has inherent number
and gender, so it compiles into four concrete categories:
\itemSpa{sg}{masc}, \itemSpa{sg}{fem}, \itemSpa{pl}{masc} and
\itemSpa{pl}{fem}, each of them containing one string. Secondly,
\t{CN} has only gender as an inherent feature, and number is
variable. Thus on the \pmcfg{} level, the Spanish \t{CN} compiles
into two concrete categories: \kind{masc} and \kind{fem}, each of them
a vector of two strings (singular and plural). Finally, \t{AP}
needs to agree in number and gender with its head, but it has its
position as an inherent feature.  Thus \t{AP} compiles into two
concrete categories: \quality{pre} and \quality{post}, each of them a
vector of four strings. 
% ---the
% combinations of \{\stackanchor{\tt \small sg}{\tt \small pl}\}
% $\times$ \{ \stackanchor{\tt \small masc}{\tt \small fem}\}

\subsubsection{Concrete functions}
Just like categories, each syntactic function from the original
grammar turns into multiple syntactic functions into the
\pmcfg{}---one for each combination of parameters of its arguments.

\begin{itemize}
\item \mod{pre}{fem~~} \t{:} \quality{pre~} $\rightarrow$ \kind{fem~} $\rightarrow$ \kind{fem}
\item  \mod{post}{fem~} \t{:} \quality{post} $\rightarrow$ \kind{fem~} $\rightarrow$ \kind{fem}
\item  \mod{pre}{masc~~}\t{:} \quality{pre~} $\rightarrow$ \kind{masc} $\rightarrow$ \kind{masc}
\item \mod{post}{masc} \t{:} \quality{post} $\rightarrow$ \kind{masc} $\rightarrow$ \kind{masc}
\end{itemize}


\subsubsection{Coercions}
\label{sec:Coercions}
As we have seen, \t{AP} in Spanish compiles into \quality{pre} and
\quality{post}. However, the difference of position is meaningful only when the
adjective is modifying the noun: ``la \emph{buena} pizza'' vs. ``la pizza
\emph{vegana}''. But when we use an adjective in a predicative position, both
classes of adjectives behave the same: ``la pizza es \emph{buena}''
and ``la pizza es \emph{vegana}''. As an optimization strategy, the
grammar creates a {\it coercion}: both \quality{pre} and \quality{post}
may be treated as \quality{*} when the distinction doesn't matter. 
Furthermore, the function \t{Pred : NP -> AP -> S} uses
the coerced category \quality{*} as its second argument, and thus
expands only into 4 variants, despite there being 8 combinations of
\t{NP}$\times$\t{AP}.

\begin{itemize}
\item \pred{sg}{fem}{*~} \t{:} \itemSpa{sg}{fem~} $\rightarrow$ \quality{*} $\rightarrow$ \comment
\item  \pred{pl}{fem}{*~} \t{:} \itemSpa{pl}{fem~} $\rightarrow$ \quality{*} $\rightarrow$ \comment
\item  \pred{sg}{masc}{*} \t{:} \itemSpa{sg}{masc} $\rightarrow$ \quality{*} $\rightarrow$ \comment
\item \pred{pl}{masc}{*} \t{:} \itemSpa{pl}{masc} $\rightarrow$ \quality{*} $\rightarrow$ \comment
\end{itemize}

\section{Generating the test suite}
\label{sec:testing}

We now have all building blocks for creating a representative and
minimal set of test cases.
In the previous section, we saw how a single abstract category
compiles into multiple concrete categories, depending on the
combinations of parameters. This compilation step can dramatically
increase the number of categories of the grammar, but it also removes
the need for dealing with these parameters explicitly when we generate
test cases. 

We choose to take one syntactic function as the base for one set of
test cases. For lexical categories, it also makes sense to
test the whole category: ``generate all trees that show that the
category \t{AP} is correctly defined''. However, we only explain in
detail the method with one syntactic function as a base.


% We chose to take one syntactic function as a base for
% Since every syntactic function also expands into multiple
% versions of itself, we took a syntactic function as the natural
% building block for a single test case.

% We now describe the generation of test cases for a given syntactic
% function.

We assume that all test cases are trees with the same start
(top-level) category, such as \t{S} in our example grammar. The
requirement is that the start category is linearized as one string only. 

\subsubsection{Enumerate functions} As we explained before, each syntactic
function turns into multiple versions, one for each combination of
parameters of its arguments. We test each of these versions
seperately. Each concrete syntactic function may produce one or several trees.
%This enumeration is the main reason we see several test cases in the examples in Section~\ref{sec:gf}.

In order to construct trees that use the syntactic function, we need
to supply it with \emph{arguments}, as well as put the resulting tree
into a \emph{context} that produces a tree in the correct start
category.

\subsubsection{Enumerate arguments} Some syntactic functions are
simply a single lexical item (for example the word \emph{good}); in
this case just the tree \t{Good} is our answer.
If we choose a function with arguments, such as \t{Pred}, then we have
to supply it with argument trees. Each argument tree needs to be a
tree belonging to the right category (in the example, \t{NP} and
\t{AP}, respectively). 

When we test a function, we want to see whether or not it uses the
right information from its arguments, in the right way. The
information that a syntactic function uses is any of the strings that
come from linearizing its arguments. In order to be able to see which
string in the result comes from which string from which arguments, we
want to generate test cases that only contain unique strings (no
duplicates). 

For example, when we test the predication function, we want to pick a
\t{AP} that actually has different forms (unique strings) for 
different genders and numbers, rather than having identical forms,
because the human would not be able to see if the \t{Pred} function
picked the wrong form. Thus a word like \t{Good} ``bueno/buena'' is
better than \t{Warm} ``caliente/caliente'', because the latter is
invariable for gender.

It is often possible to generate one combination of arguments where
all strings in the linearizations are different. However, it is not
always possible to do this, which is why we in general aim to generate
a set of combinations of arguments, where for each pair of strings
from the arguments, there is always one test case where those strings
are different. In this way, if the syntactic function contains a
mistake, there is always one test case that reveals it.

\paragraph{Example: Test cases using \t{Mod}} Let us test the function
\t{Mod : AP $\rightarrow$ CN $\rightarrow$ CN} in the Spanish
concrete syntax.
Firstly, we need a minimal and representative set of arguments:
one premodifier and one postmodifier \t{AP} (\t{Good} and
\t{Vegan}), as well as one feminine and one masculine
\t{CN} (\t{Pizza} and \t{Wine}). Now, our full set of test cases are
\t{Mod} applied to the cross product of \{\stackanchor{\tt \small
  Good}{\tt \small Vegan}\} $\times$ \{\stackanchor{\tt \small
  Pizza}{\tt \small Wine}\}, as seen in Table~\ref{tab:adjAttr}.

\begin{table}
\caption{Agreement and placement of adjectives in attributive position}
\label{tab:adjAttr}
\centering
\begin{tabular}{| l | l |}
\hline
\t{Mod Good Pizza}   & \t{Mod Good Wine} \\ 
\textsc{(sg)} buena pizza             & \textsc{(sg)} buen vino \\
\textsc{(pl)} buenas pizzas           & \textsc{(pl)} buenos vinos \\ \hline

\t{Mod Vegan Pizza}   & \t{Mod Vegan Wine} \\ 
\textsc{(sg)} pizza  vegana            & \textsc{(sg)} vino  vegano \\
\textsc{(pl)} pizzas  veganas          & \textsc{(pl)} vinos  veganos \\ \hline
\end{tabular}
\end{table}

\subsubsection{Enumerate contexts} The third and last enumeration we perform when generating test cases is to generate all possible \emph{uses} of a function. After we provide a function with arguments, we need to put the result into a context, so that we can generate a single string from the result (a sentence). We do this for all trees we have generated so far.

The important thing here is that the generated set of contexts shows
all the possible different ways the tree can be used. For example, for
a test tree with an inflection table of 4 forms, we would generate 4 different sentences in which each of the 4 inflections is used.

By \emph{context}, we mean a tree in the start category, with a
\emph{hole} of type \t{CN}. Since \t{CN} is variable for
number, we need two contexts:
% A tree of
% type \t{CN} can be plugged into the hole to form a tree in the start
% category \t{S}. There is only one function, \t{Pred : NP ->
%   AP -> S}, that constructs a \t{S}. 
% We cannot give a \t{CN} directly as an argument to \t{Pred}, but we
% do have four functions that turn a \t{CN} into an \t{NP}, namely
% \t{This}, \t{That}, \t{These} and \t{Those}. However, we only need two
% of them, 
one that picks out the singular form and other that picks out
the plural form. This suggests that we should apply two different
\t{CN -> NP} functions, for instance \t{This} and \t{These}, and
give their results to the \t{Pred} function, which constructs a \t{S}.
In contrast, the second argument to \t{Pred} doesn't make a difference
in what form we pick out of the \t{CN}---we just want to pick
something that has maximally different forms, so the program is sure
not to pick \t{Warm}, which is invariable for gender. By random
selection, let us pick \t{Italian}.
The final contexts are \verb|Pred (This _) Italian| and \verb|Pred (These _) Italian|.
We insert the 4 test cases from Figure~\ref{tab:adjAttr} into the
holes, and get 8 trees in total: 

\begin{table}
\caption{Complete test cases to test \t{Mod}}
\label{tab:testCases}
\centering
\begin{tabular}{| l | l |}
\hline
\t{Pred (This (Mod Good Pizza)) Italian} & \t{Pred (This (Mod Good Wine))
                                        Italian} \\ 
esta buena pizza es italiana          & este buen vino es italiano \\ \hline
\t{Pred (These (Mod Good Pizza)) Italian} & \t{Pred (These (Mod Good Wine))
                                        Italian} \\ 
estas buenas pizzas son italianas          & estos buenos vinos son italianos \\ \hline
\t{Pred (This (Mod Vegan Pizza)) Italian} & \t{Pred (This (Mod Vegan Wine))
                                        Italian} \\ 
esta pizza vegana es italiana          & este vino vegano es italiano \\ \hline
\t{Pred (These (Mod Vegan Pizza)) Italian} & \t{Pred (These (Mod Vegan Wine))
                                        Italian} \\ 
estas pizzas veganas son italianas          & estos vinos veganos son italianos \\ \hline
\end{tabular}
\end{table}

\todo{Write this better?} So, what we want to compute is, given the result category T of the syntactic function, and the start category S of the grammar, a minimal set of contexts in the start category S with hole of type T, such that any string appearing in the linearization of T also appears somewhere in the linearization of S. We compute this by setting up a system of equations for each category C in the grammar: for each C, we define all the relevant contexts with hole type C in terms of all the relevant contexts with hole type C' for other categories C' that use C. So, the answer for each category is expressed in terms of the answer for other categories. In general, this system of equations is \emph{recursive}, and we use a fixpoint iteration to compute the smallest solution.

\subsubsection{Pruning the trees within one set of test cases} 
On the scope of our tiny example grammar, this pruning method is
easiest to illustrate when we test a category instead of a function;
however, in bigger grammars, the need arises when testing functions as
well. 
In order to test the category \t{AP}, we need in total 12 example sentences:
\begin{itemize}
\item 4 test cases for a premodifier \t{AP} as modifier;
\item 4 test cases for a postmodifier \t{AP} as modifier;
\item 4 test cases for \emph{any} \t{AP} as predicative.
\end{itemize}
If we did not prune the trees, the program would generate
redundant test cases with \t{AP}s in predicative position: 4 for
premodifier and 4 for postmodifier adjectives,
As explained in Section~\ref{sec:Coercions}, the grammar detects
these redundancies automatically, so we just leverage the coercions 
already existing in the grammar.

% In our example we tested explicitly
% \t{Mod}, but suppose we want to test the function \t{Very} instead. On
% the \pmcfg{} level, it compiles into two concrete functions,
% \very{pre} \t{:} \quality{pre} $\rightarrow$ \quality{pre} and 
% \very{post} \t{:} \quality{post} $\rightarrow$ \quality{post}.
% By enumerating the arguments, we end up with the test cases \t{Very
%   Good} and \t{Very Vegan}, and each of them has to be placed into
% contexts. 

\subsubsection{Pruning the trees to test the whole grammar}
So far we have completely ignored that one tree can test more
than one function. In fact, the 8 test sentences created for \t{Mod}
happen to also test \t{Pred} exhaustively.
Let us recap the steps we took to create them for \t{Mod}:
enumerating arguments brought us \t{Good}, \t{Vegan}, \t{Pizza} and
\t{Wine}, and enumerating contexts brought us \t{This} and
\t{These}. Had we been creating test cases for \t{Pred}, we would've
gotten \t{This} and \t{These} at the stage of enumerating arguments,
and then there would've been no need for contexts, because \t{Pred}
already creates the start category \t{S}.

There is a simple way to detect the redundancy: make the generation of
arguments completely deterministic, e.g. always choose the function
that is alphabetically first. The downside is that we get a lot of
redundancy, in the style of ``the good pizza is good''. The problem
gets more severe when testing functions with more arguments: ``the
pizza gives the pizza the pizza'' is not only boring but confusing.
However, if we want to generate sentences for the whole grammar at one 
go, we can split the generation in two stages: first stage is
deterministic, where every feminine noun is \t{Pizza}, and we can 
eliminate redundancies by just eliminating copies of the same
tree. Then, when we have a set of unique trees, we can substitute
individual words in them with other words in the same concrete
category: ``the house gives the pizza the fountain'' has the same
testing property as the version with pizza in every role, but at least
it is easier to keep track who does what, and compare the translations
of the same tree. 

It would be ideal to generate sentences that make sense,
such as ``the waitress gives the girl the pizza''. If the grammar is
purely syntactic, we would need external tools to ensure semantic
coherence, but if the grammatical categories already include semantic
distinctions, e.g. limiting the subject and indirect object of
\emph{give} into humans, that naturally restricts the generated test suite.

\section{Use cases}

Here is a typical use for the tool. 
Let us take the GF resource grammar \cite{ranta2009rgl} for Spanish, and
pick the function \t{AdvCN : Adv -> CN -> CN}, modifies a \t{CN} with
an adverb.
The tool generates test cases in the way described previously,
which include the following (slightly simplified) trees: 
\begin{itemize}
\item \t{AdvCN (PrepNP next\_to (DetNP your)) hill} `hill next to
yours'
\item \t{AdvCN (PrepNP next\_to (DetNP your)) house} `house next
to yours'
\end{itemize}
In Spanish, the words \emph{hill} and \emph{house} have different
genders, and the word \emph{yours} has to agree in gender 
with the antecedent: \emph{(la) casa al lado de la tuya} and \emph{(el)
  cerro al lado del tuyo}. The test cases reveal a bug, where \t{DetNP your} 
picks a gender too soon, say always masculine, instead of leaving it
open in an inflection table. We implement a fix by adding gender as a
parameter to the \t{Adv} type, and have \t{AdvCN} choose the correct
form based on the gender of the \t{CN}.

After implementing the fix, we run a second test case generation: this
time, not meant for human eyes, but just to compare the old and new
versions of the grammar. We want to make sure that our changes 
have not caused new bugs in other functions. The simplest strategy is
to generate test cases for \emph{all} functions in both grammars, and
only show those outputs that differ between the grammars. After our
fixes, we get the following differences: 

\begin{itemize}
\item \t{DetCN the (AdvCN (PrepNP next\_to (DetNP your)) house)}
  \begin{itemize}
   \item \emph{\sout{la casa al lado {\bf  del tuyo}}}
   \item \emph{la casa al lado {\bf  de la tuya}}
  \end{itemize}
\item \t{DetCN the (AdvCN (PrepNP without (DetNP this)) house)}
  \begin{itemize}
   \item \emph{\sout{la casa sin {\bf  esto}}}
   \item \emph{la casa sin {\bf esta}}
  \end{itemize}
\end{itemize}

\noindent We notice a side effect that we may not have thought of: the
gender is retained in all adverbs made of NPs made of determiners, so
now it has become impossible to say ``the hill without \emph{that}'' and
pointing to a house. So we do another round of modifications, compute
the difference (to the original grammar or to the intermediate), and
see if something else broke.

\section{Evaluation}

The previous section describes the use of the tool for developing and
testing one function at a time. However, it is interesting to see 

% \begin{table}
% \caption{Table captions should be placed above the
% tables.}\label{tab1}
% \begin{tabular}{|l|l|l|}
% \hline
% Heading level &  Example & Font size and style\\
% \hline
% Title (centered) &  {\Large\bfseries Lecture Notes} & 14 point, bold\\
% 1st-level heading &  {\large\bfseries 1 Introduction} & 12 point, bold\\
% 2nd-level heading & {\bfseries 2.1 Printing Area} & 10 point, bold\\
% 3rd-level heading & {\bfseries Run-in Heading in Bold.} Text follows & 10 point, bold\\
% 4th-level heading & {\itshape Lowest Level Heading.} Text follows & 10 point, italic\\
% \hline
% \end{tabular}
% \end{table}


% \noindent Displayed equations are centered and set on a separate
% line.
% \begin{equation}
% x + y = z
% \end{equation}
% Please try to avoid rasterized images for line-art diagrams and
% schemas. Whenever possible, use vector graphics instead (see
% Fig.~\ref{fig1}).

% \begin{figure}
% \includegraphics[width=\textwidth]{fig1.eps}
% \caption{A figure caption is always placed below the illustration.
% Please note that short captions are centered, while long ones are
% justified by the macro package automatically.} \label{fig1}
% \end{figure}

% \begin{theorem}
% This is a sample theorem. The run-in heading is set in bold, while
% the following text appears in italics. Definitions, lemmas,
% propositions, and corollaries are styled the same way.
% \end{theorem}
% %
% % the environments 'definition', 'lemma', 'proposition', 'corollary',
% % 'remark', and 'example' are defined in the LLNCS documentclass as well.
% %
% \begin{proof}
% Proofs, examples, and remarks have the initial word in italics,
% while the following text appears in normal font.
% \end{proof}
% For citations of references, we prefer the use of square brackets
% and consecutive numbers. Citations using labels or the author/year
% convention are also acceptable. The following bibliography provides
% a sample reference list with entries for journal

%
% ---- Bibliography ----
%
% BibTeX users should specify bibliography style 'splncs04'.
% References will then be sorted and formatted in the correct style.
%
\bibliographystyle{splncs04}
\bibliography{../coling2018/bibliography}

\end{document}
